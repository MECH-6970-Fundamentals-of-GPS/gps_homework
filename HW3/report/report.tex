\documentclass{article}
\usepackage{graphicx}
\usepackage{amsmath}
\usepackage{cleveref}
\usepackage{float}

\title{GPS - HW3}
\author{Walter Livingston}
\date{March 2024}

\begin{document}

\maketitle

\section*{Problem I}
This problem is to look at noise models.

\subsection*{Part A}
Show mathematically that the error from the sum OR difference of two independent random measurements created from $y = 3a \pm 4b$, where a and b are both zero mean with variance $\sigma_a^2$ and $\sigma_b^2$ results in:
    \begin{center}
    $\mu_y = 0$ and $\sigma_y = \sqrt{9\sigma_a^2 + 16\sigma_b^2}$
    \end{center}
Perform a 1000 run Monte-Carlo simulation to verify your results.  Plot a histogram of the Monte-Carlo simulation to verify the output y is Gaussian.

\subsection*{Solution}
The mean of y is the expectation of y.  Following this algebraically, $\mu_y$ is
\begin{gather*}
    E \left[y\right] = E \left[3a+4b\right]\\
    E \left[y\right] = E \left[3a\right] + E \left[4b\right]\\
    E \left[y\right] = 3E \left[a\right] + 4 E \left[b\right]\\
    \mu_y = 3\mu_a + 4\mu_b = 0 \\
\end{gather*}
The standard deviation of y is the square root of the expectation of $(y - \overline{y})^2$.  Following this algebraically, $\sigma_y$ is
\begin{gather*}
    E \left[(y - \overline{y})^2\right] = E \left[y^2\right]\\
    E \left[y^2\right] = E \left[(3a + 4b)^2\right]\\
    E \left[y^2\right] = E \left[3a^2 + 48ab + 4b^2\right]\\
    E \left[y^2\right] = 9E \left[a^2\right] + 48E\left[ab\right] + 16E\left[b^2\right]\\
    E \left[y^2\right] = 9E \left[a^2\right] + 48E\left[a\right]E\left[b\right] + 16E\left[b^2\right]\\
    \sigma_y^2 = 9\sigma_a^2 + 48\mu_a\mu_b + 16\sigma_b^2\\
    \sigma_y = \sqrt{9\sigma_a^2 + 16\sigma_b^2}
\end{gather*}
The expectation of $ab$ can be separated as they are uncorrelated, and then the term reduces to zeros as the variables are zero mean.  Also, the expectation of $a^2$ and $b^2$ is equal to their variance as they are zero mean.\\
The figure below shows the histogram of the Monte-Carlo Simulation.
\begin{figure}[H]
    \centering
    \includegraphics[width=0.75\linewidth]{../figures/p1_histogram.png}\label{fig:p1_hist}
    \caption{Histogram of the Monte Carlo Simulation}
\end{figure}
\section*{Problem II}
Perform a 1000 run Monte-Carlo simulation (of 10 minutes) to look at the error growth of a
random walk (integrated white noise). Use a white noise with 1-sigma value of 0.1 and 0.01
and compare the results. Plot the mean and standard deviation of the Monte-Carlo simulation
along with one run of the simulation (show that the random walk is zero mean with a
standard deviation is $\sigma_{\int w} = \sigma_w \Delta t\sqrt{k} = \sigma_w\sqrt{t \Delta t}$ (where k is the sample number).

\subsection*{Solution}
\begin{figure}[H]
    \centering
    \includegraphics[width=0.75\linewidth]{../figures/p2_rw_mean.png}\label{fig:p2_rw_mean}
    \caption{Mean of the Monte Carlo Simulation of Random Walk}
\end{figure}

\begin{figure}[H]
    \centering
    \includegraphics[width=0.75\linewidth]{../figures/p2_rw_sigma.png}\label{fig:p2_rw_sigma}
    \caption{Standard Deviation of the Monte Carlo Simulation of Random Walk}
\end{figure}

\begin{figure}[H]
    \centering
    \includegraphics[width=0.75\linewidth]{../figures/p2_rw_run.png}\label{fig:p2_rw_run}
    \caption{Integrated White Noise vs. Time (Run 1 of Monte Carlo)}
\end{figure}

\subsection*{Bonus}
Provide a Histogram of the Monte-Carlo simulation at a few select time slots.

\subsection*{Solution}
\begin{figure}[H]
    \centering
    \includegraphics[width=0.75\linewidth]{../figures/p2_bonus.png}\label{fig:p2_bonus}
    \caption{Histograms of Individual Time Points}
\end{figure}

\subsection*{Part B}
Perform a 1000 Monte-Carlo simulation to look at the error growth of a 1st order Markov
process (integrated filtered noise) of the form $\dot{x} = -\frac{1}{\tau}x+w$. Use the same noise characteristics as above and compare the results with a 1 second and 100 second time
constant (this results in 4 combinations). Comment on how changing the time constant
and changing the standard deviation of the noise effects the error. Show that the 1st order
Markov process is zero mean with a standard deviation of is $\sigma_x = \sigma_w\Delta t\sqrt{\frac{A^{2t}-1}{A^2-1}}$ where $A = (1-\frac{\Delta t}{\tau})$. Note that for a positive time constant (i.e. stable system) the standard deviation has a steady-state value.

\subsection*{Solution}
\begin{figure}[H]
    \centering
    \includegraphics[width=0.75\linewidth]{../figures/p2_gm_mean.png}\label{fig:p2_gm_mean}
    \caption{Mean of the Monte Carlo Simulation of Gauss Markov}
\end{figure}

\begin{figure}[H]
    \centering
    \includegraphics[width=0.75\linewidth]{../figures/p2_gm_sigma.png}\label{fig:p2_gm_sigma}
    \caption{Standard Deviation of the Monte Carlo Simulation of Gauss Markov}
\end{figure}

As seen from the above figures, increasing $\tau$ increases both the rise time and the steady state of $\sigma$.

\section*{Problem III}
Determine the expected uncertainty for an L1-L5 ionosphere free pseudorange measurement
and L2-L5 ionosphere free pseudorange. Assuming all measurements have the same
accuracy (L1, L2, L5) which will provide the best ionosphere estimate?
\subsection*{Solution}
The equation for a pseudorange can be found in \Cref{eq:pseudorange}

\begin{equation}
    \rho = r + c\delta t + I + M + T + \nu_{rcvr}\label{eq:pseudorange}
\end{equation}

The only non-deterministic part of \Cref{eq:pseudorange} is the receiver noise, $\nu_{rcvr}$.  The equation for an L1+L2 Ionospheric-Free pseudorange is shown in \Cref{eq:IF_pseudorange}

\begin{equation}
    \rho_{IF} = \frac{f_{L1}^2}{f_{L1}^2 - f_{L2}^2}\rho_{L1} + \frac{f_{L2}^2}{f_{L1}^2 - f_{L2}^2}\rho_{L2}\label{eq:IF_pseudorange}
\end{equation}

If we subtract off the pseudorange itself, and consider the fact that the receiver noise is the only non-determinant part of the pseudorange equation, we can turn \Cref{eq:IF_pseudorange} into the following:

\begin{equation}
    \delta \rho_{IF} = \frac{f_{L1}^2}{f_{L1}^2 - f_{L2}^2}\nu_{rcvr} + \frac{f_{L2}^2}{f_{L1}^2 - f_{L2}^2}\nu_{rcvr}\label{eq:IF_del_psr}
\end{equation}

The receiver noises are not frequency-dependent, so given the signals are read with the same receiver, the noises would be equal.  Following \Cref{eq:IF_del_psr} through algebraically, the $\sigma$ on $\delta \rho$ is:

\begin{gather*}
    E \left[(\delta\rho_{IF})^2\right] = E \left[(\frac{f_{L1}^2}{f_{L1}^2 - f_{L2}^2}\nu_{rcvr} + \frac{f_{L2}^2}{f_{L1}^2 - f_{L2}^2}\nu_{rcvr})\right]\\
    E \left[(\delta\rho_{IF})^2\right] = \frac{f_{L1}^2}{f_{L1}^2 - f_{L2}^2}E \left[\nu_{rcvr}\right] + \frac{f_{L2}^2}{f_{L1}^2 - f_{L2}^2}E \left[\nu_{rcvr}\right]\\
    \sigma_{\rho_{IF}}^2 =  \frac{f_{L1}^2}{f_{L1}^2 - f_{L2}^2}\sigma_{rcvr}^2 + \frac{f_{L2}^2}{f_{L1}^2 - f_{L2}^2}\sigma_{rcvr}^2\\
    \sigma_{\rho_{IF}}^2 =  \sigma_{rcvr}^2(\frac{f_{L1}^2}{f_{L1}^2 - f_{L2}^2} + \frac{f_{L2}^2}{f_{L1}^2 - f_{L2}^2})\\
    \sigma_{\rho_{IF}} =  \sqrt{\sigma_{rcvr}^2(\frac{f_{L1}^2}{f_{L1}^2 - f_{L2}^2} + \frac{f_{L2}^2}{f_{L1}^2 - f_{L2}^2})}\\
\end{gather*}

Using the above equation for the standard deviation on the Ionospheric-Free pseudorange, an estimate of that value can be calculated for any combination of frequencies, 
just replacing $f_{L1}$ and $f_{L2}$ with the utilized frequencies.  Doing this shows that L1+L5 is the best combination for obtaining Ionospheric-Free psuedoranges as it 
produces the smallest scale factor to the standard deviation.\\
A table of the scale values on $\sigma$ is shown below for the different frequency combinations.

\begin{center}
    \begin{tabular}{ | l | c | r }
      \hline
      Frequencies & Scale for $\sigma$ \\ \hline
      L1+L2 & 2.9793 \\ \hline
      L1+L5 & 2.5840 \\ \hline
      L2+L5 & 16.4918 \\ \hline
    \end{tabular}
  \end{center}

\section*{Problem IV}
Show that the differential GPS problem is linear. In other words derive the following
expression:
    \begin{center}
    $\Delta\rho=\begin{bmatrix}
        -uv_x & -uv_y & -uv_z & 1
        \end{bmatrix}\begin{bmatrix}
        r_x\\r_y\\r_z\\c\delta t_{ab}
        \end{bmatrix}$
    \end{center}
\subsection*{Solution}

To show the problem of differential GPS is linear, it begins with the nonlinear, matrix equation for the user's pseudorange as shown in \Cref{eq:psr_mat}

\begin{equation}
    \delta\rho_u = \begin{bmatrix}
        -uv_x & -uv_y & -uv_z & 1
        \end{bmatrix}\begin{bmatrix}
        \delta x_u\\\delta y_u\\\delta z_u\\c\delta t_u
        \end{bmatrix}\label{eq:psr_mat}
\end{equation}

Given that the base and the user are close enough, the unit vectors for the base and the user can be assumed to be the same.  
The user psuedorange, as shown above, can be differenced with a similar looking equation for the base resulting in the below equation.

\begin{equation}
    \Delta\rho = \begin{bmatrix}
        -uv_x & -uv_y & -uv_z & 1
        \end{bmatrix}\begin{bmatrix}
        \delta x_u - \delta x_b\\\delta y_u - \delta y_b\\\delta z_u - \delta z_b\\c\delta t_u - c\delta t_b
        \end{bmatrix}\label{eq:del_psr_mat}
\end{equation}

Because the unit vectors for the base and the user are assumed to be the same, they are factored out.  Now, given that $\delta x = x - \hat{x}$ and the assumption that the base and user 
pseudorange equations are linearized around the same point($\hat{x_u} = \hat{x_b}$), the above equation reduces to:

\begin{center}
$\Delta\rho=\begin{bmatrix}
    -uv_x & -uv_y & -uv_z & 1
    \end{bmatrix}\begin{bmatrix}
    r_x\\r_y\\r_z\\c\delta t_{ub}
    \end{bmatrix}$
\end{center}
where $r_x = x_u - x_b$.

\section*{Problem V}
Set up your own 2D planar trilateration problem. Place the SVs at (0,300) (100,400),
(700,400), and (800,300). Generate a range measurement for a base station at (400,0) and a
user at (401,0).

\subsection*{Part A}
Solve for the position of the user using 2 SVs and then 4 SVs assuming no clock
errors. How does the PDOP change for the two cases?

\subsection*{Part A Comments}
The PDOP was $\sim$5x larger with only 2 SVs as compared to 4.

\subsection*{Part B}
Solve for the position of the user assuming you need to solve for the user clock bias.
What is the PDOP with all 4 satellites.

\subsection*{Part C}
Calculate a differential solution between the base and user using a single difference
model and assuming you must solve for a clock bias between the base station and
user. What is the PDOP with all 4 satellites?

\subsection*{Part D}
Calculate a differential solution between the base and user using a double difference
model to remove the clock bias between the base station and user. What is the PDOP
with all 4 satellites?

\subsection*{Part E}
Assuming the range error is zero mean with unit variance, what is the order of
accuracy in the above 4 solution methods?

\subsection*{Solution}

\begin{center}
    \begin{tabular}{ | l | c | c | c |}
        \hline
        Scenario & x & y & b \\ \hline
        A-2SVs & 401 & $\sim$0 & - \\ \hline
        A-4SVs & 401 & $\sim$0 & - \\ \hline
        B-4SVs + Bias & 401 & $\sim$0 & $\sim$0 \\ \hline
        C-DGPS-Single & 401 & -0.0014 & -0.0005 \\ \hline
        D-DGPS-Double & 401 & -0.0014 & - \\ \hline
    
    \end{tabular}
  \end{center}

\begin{figure}[H]
    \centering
    \includegraphics[width=0.75\linewidth]{../figures/p5_pdop.png}\label{fig:p5_pdop}
    \caption{PDOP \& Error for Varying Circumstances}
\end{figure}

\section*{Problem VI}
(Bonus for Undergrads/Required for Grads). Repeat problem \#4 using 4 and 8 SV positions
from Lab \#2 (4,7,8,9,16,21,27,30). Comment on any difference or similarities with the
planar problem in \#3.

\subsection*{Solution}
\begin{figure}[H]
    \centering
    \includegraphics[width=0.75\linewidth]{../figures/p6_4svs_pdop.png}\label{fig:p6_4svs_pdop}
    \caption{PDOP for Real Data w/ 4SVs}
\end{figure}

\begin{figure}[H]
    \centering
    \includegraphics[width=0.75\linewidth]{../figures/p6_8svs_pdop.png}\label{fig:p6_8svs_pdop}
    \caption{PDOP for Real Data w/ 8SVs}
\end{figure}

Though not exact, the 4SV scenario with the real satellite data has a PDOP ~3x larger than the 8SV scenario.  This is a similar 
difference to what was shown between the 4SV and 2SV scenarios in Part V.

\section*{Problem VII}
Chapter 2, Problem 1a and 1b for PRN \#4. Repeat 1a for PRN \#7

\subsection*{Solution}
\begin{figure}[H]
    \centering
    \includegraphics[width=0.75\linewidth]{../figures/p7_prn4_16.png}\label{fig:p7_prn4_16}
    \caption{First 16 bits of C/A Code for PRN 4}
\end{figure}
\begin{figure}[H]
    \centering
    \includegraphics[width=0.75\linewidth]{../figures/p7_prn7_16.png}\label{fig:p7_prn7_16}
    \caption{First 16 bits of C/A Code for PRN 7}
\end{figure}

\section*{Problem VIII}
Using your PRN sequence for PRN 4 and 7, repeat problem \#2 from HW\#1. Compare the
results to the results for your made up sequence.

\subsection*{Solution}
The results from HW1 - Problem II greatly resemble the results of the same analysis done on the PRN signals.  This is indicative of the fact that the PRNs contain a lot of the same properties as a random sequence which makes sense given they are named, "Pseudo Random Number."
\subsection*{HW1 - Problem II Results}
\begin{figure}[H]
    \centering
    \includegraphics[width=0.75\linewidth]{../figures/p8_hw1_2a.png}\label{fig:p8_hw1_2a}
    \caption{HW1 Histogram of Random Sequences}
\end{figure}
\begin{figure}[H]
    \centering
    \includegraphics[width=0.75\linewidth]{../figures/p8_hw1_2b.png}\label{fig:p8_hw1_2b}
    \caption{HW1 PSD of Random Sequences}
\end{figure}
\begin{figure}[H]
    \centering
    \includegraphics[width=0.75\linewidth]{../figures/p8_hw1_2c.png}\label{fig:p8_hw1_2c}
    \caption{HW1 Autocorrelation of Random Sequences}
\end{figure}
\begin{figure}[H]
    \centering
    \includegraphics[width=0.75\linewidth]{../figures/p8_hw1_2d.png}\label{fig:p8_hw1_2d}
    \caption{HW1 Cross Correlation of Random Sequences}
\end{figure}
\begin{figure}[H]
    \centering
    \includegraphics[width=0.75\linewidth]{../figures/p8_hw1_2bonus1.png}\label{fig:p8_hw1_2bonus1}
    \caption{HW1 Cross Correlation of Long Random Sequences}
\end{figure}
\begin{figure}[H]
    \centering
    \includegraphics[width=0.75\linewidth]{../figures/p8_hw1_2bonus2.png}\label{fig:p8_hw1_2bonus2}
    \caption{HW1 Cross Correlation of Long Random Sequences}
\end{figure}

\subsection*{Part A}
Plot the histogram on each sequence

\subsection*{Solution}
\begin{figure}[H]
    \centering
    \includegraphics[width=0.75\linewidth]{../figures/p8_prn4_hist.png}\label{fig:p8_prn4_hist}
    \caption{PRN 4 Histogram }
\end{figure}

\begin{figure}[H]
    \centering
    \includegraphics[width=0.75\linewidth]{../figures/p8_prn7_hist.png}\label{fig:p8_prn7_hist}
    \caption{PRN 7 Histogram }
\end{figure}

\subsection*{Part B}
Plot the spectral analysis on each sequence

\subsection*{Solution}
\begin{figure}[H]
    \centering
    \includegraphics[width=0.75\linewidth]{../figures/p8_prn4_psd.png}\label{p8_prn4_psd}
    \caption{PRN 4 Power Spectral Density}
\end{figure}
\begin{figure}[H]
    \centering
    \includegraphics[width=0.75\linewidth]{../figures/p8_prn7_psd.png}\label{p8_prn7_psd}
    \caption{PRN 7 Power Spectral Density}
\end{figure}

\subsection*{Part C}
Plot the auto correlation each sequence with itself (i.e. a sequence delay cross
correlation)

\subsection*{Solution}
\begin{figure}[H]
    \centering
    \includegraphics[width=0.75\linewidth]{../figures/p8_prn4_acorr.png}\label{p8_prn4_acorr}
    \caption{PRN 4 Autocorrelation}
\end{figure}
\begin{figure}[H]
    \centering
    \includegraphics[width=0.75\linewidth]{../figures/p8_prn7_acorr.png}\label{p8_prn7_acorr}
    \caption{PRN 7 Autocorrelation}
\end{figure}

\subsection*{Part D}
Plot the cross auto correlation between the two sequences

\subsection*{Solution}
\begin{figure}[H]
    \centering
    \includegraphics[width=0.75\linewidth]{../figures/p8_prn47_xcorr.png}\label{p8_prn47_xcorr}
    \caption{PRN 4 \& 7 Cross Correlation}
\end{figure}

\section*{Problem IX}
Take your C/A code from problem above (i.e. PRN \#4 and \#7) and multiply it times the L1
Carrier (your C/A code must be in the form -1 and +1). Perform a spectral analysis
(magnitude) on the resultant signal. You will need to make sure to “hold” your C/A code bits
for the correct length of time (I suggest using a sample rate 10x the L1 carrier frequency –
meaning each chip of the C/A code will be used for 10 samples of the sine wave).

\subsection*{Solution}
\begin{figure}[H]
    \centering
    \includegraphics[width=0.75\linewidth]{../figures/p9_prn4_psd.png}\label{p9_prn4_psd}
    \caption{C/A Code for PRN 4 + L1 Carrier PSD}
\end{figure}

\begin{figure}[H]
    \centering
    \includegraphics[width=0.75\linewidth]{../figures/p9_prn4_psd.png}\label{p9_prn4_psd}
    \caption{C/A Code for PRN 4 + L1 Carrier PSD}
\end{figure}

\end{document}
