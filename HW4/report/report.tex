\documentclass{article}
\usepackage{graphicx}
\usepackage{amsmath}
\usepackage{cleveref}
\usepackage{matlab-prettifier}
\usepackage{float}

\title{GPS - HW3}
\author{Walter Livingston}
\date{March 2024}

\begin{document}

\maketitle

\section*{Problem I}
Write a controller (matlab or Simulink) for a simple 1/s plant. Assume a unit step input for
the reference, r(t).
\subsection*{Part A}
What type of controller did you use. Provide the Gain Margin, Phase Margin, closed-
loop eigenvalues, and steady state error.
\subsection*{Solution}
I used a proportional controller.  The characteristics of my controller are in the table below.  Plots shown in C.

\begin{center}
    \begin{tabular}{ | l | c | }
      \hline
      Gain Margin & $\infty$ \\ \hline
      Phase Margin & 90 \\ \hline
      Eigenvalues & $0 \& -2$ \\ \hline
      SS Error & $\sim 0$ \\ \hline
    \end{tabular}
\end{center}

\subsection*{Part B}
What is the steady state error if the reference $r(t)$ is a unit ramp input
\subsection*{Solution}
Steady state error is 0.5.
\subsection*{Part C}
Redesign the controller to track the ramp input and repeat Part A.
\subsection*{Solution}
The controller was redesigned as a PI controller.  The controller characteristics are listed in the table below.

\begin{center}
    \begin{tabular}{ | l | c | }
      \hline
      Gain Margin & $\infty$ \\ \hline
      Phase Margin & 90 \\ \hline
      Eigenvalues & $-1 \& -1$ \\ \hline
      SS Error & $\sim 0$ \\ \hline
    \end{tabular}
\end{center}

\begin{figure}[H]
    \centering
    \includegraphics[width=0.75\linewidth]{../figures/p1.png}\label{fig:p1}
    \caption{Problem 1 Step \& Ramp Responses}
\end{figure}

\section*{Problem II}
Take the sampled 100 Hz sine wave (generate\textunderscore signal(1)) from the website (sampled at 1
MHz, i.e. $Ts=1e-6$)
\subsection*{Part A}
Develop a simple (1 Hz) PLL to track the phase of the signal. Provide plots of phase,
phase error, as well as the estimated signal vs. true signal.
\subsection*{Solution}
\begin{figure}[H]
    \centering
    \includegraphics[width=0.75\linewidth]{../figures/p2a_phase.png}\label{fig:p2a_phase}
    \caption{Phase \& Error vs. Time for 1Hz Bandwidth}
\end{figure}

\begin{figure}[H]
    \centering
    \includegraphics[width=0.75\linewidth]{../figures/p2a_signal.png}\label{fig:p2a_signal}
    \caption{Signal \& Replica vs. Time for 1Hz Bandwidth}
\end{figure}

\subsection*{Part B}
Double the PLL bandwidth and repeat part a.
\subsection*{Solution}
\begin{figure}[H]
    \centering
    \includegraphics[width=0.75\linewidth]{../figures/p2a_phase.png}\label{fig:p2b_phase}
    \caption{Phase \& Error vs. Time for 2Hz Bandwidth}
\end{figure}

\begin{figure}[H]
    \centering
    \includegraphics[width=0.75\linewidth]{../figures/p2a_signal.png}\label{fig:p2b_signal}
    \caption{Signal \& Replica vs. Time for 2Hz Bandwidth}
\end{figure}

\subsection*{Part C}
Determine the true frequency and repeat part a.
\subsection*{Solution}

\subsection*{Part D}
Modify your PLL to work from the unknown frequency and repeat part a assuming 10
Hz signal.
\subsection*{Solution}
\begin{figure}[H]
    \centering
    \includegraphics[width=0.75\linewidth]{../figures/p2d_phase.png}\label{fig:p2d_phase}
    \caption{Phase \& Error vs. Time for 10Hz Initial Frequency}
\end{figure}

\begin{figure}[H]
    \centering
    \includegraphics[width=0.75\linewidth]{../figures/p2d_signal.png}\label{fig:p2d_signal}
    \caption{Signal \& Replica vs. Time for 10Hz Initial Frequency}
\end{figure}

\section*{Problem III}
Modify your PLL from problem \#3 to operate as a Costas loop filter. Take the 88 second
data (generate\textunderscore signal(2)) sampled at 1 MHz and decode the data message on the 100 Hz
sinusoid using your Costas loop filter. The data bits are 1 second wide and are comprised
of 8 bit ascii characters. The following functions will be of use:
\subsection*{Solution}
The databits translate to AU WarEagle.
\begin{figure}[H]
    \centering
    \includegraphics[width=0.75\linewidth]{../figures/p3_IQ.png}\label{fig:p3_IQ}
    \caption{In-Phase \& Quadrature}
\end{figure}

\section*{Problem IV}
Develop a simple DLL to phase align the sequence shown below with the digital signal
(generate\textunderscore signal(3)) provided on the website. The sequence is sampled at 10 samples per
chip. Plot the delay vs. time (or frequency vs. time depending on your implementation).
\subsection*{Solution}
\begin{figure}[H]
    \centering
    \includegraphics[width=0.75\linewidth]{../figures/p4_tau.png}\label{fig:p4_tau}
    \caption{Code Phase vs. Time}
\end{figure}

\section*{Problem VI}
Take the PRN code for SV \#4 or \#7 (i.e. from HW \#3) and upsample it such that there are
16 samples at each chip (i.e. the length of this vector will be 1023*16 long).

\subsection*{Part A}
Show the autocorrelation calculation from -5 chips to +5 chips in 1/16 chip increments.
\subsection*{Solution}
Combined Plot shown in Part B
\subsection*{Part B}
Repeat with noise $(\sigma=0.2)$ added to the non-shifted signal.\\
Recall that the autocorrelation function can be calculated as:
\begin{gather*}
    R(\tau) = x^T x(\tau) = \sum\textunderscore {1}^{16*1023}x(k)x(k+\tau)
\end{gather*}
Note that you must roll $x(\tau)$ around when shifting it, i.e. x(16 × 1023 + 10) = x(11)\\
You will need code that does the upsample and code shift for Lab \#4 (as well as the
problem below).
\subsection*{Solution}
\begin{figure}[H]
    \centering
    \includegraphics[width=0.75\linewidth]{../figures/p6_corr.png}\label{fig:p6_corr}
    \caption{Autocorrelation Comparison}
\end{figure}

\subsection*{Problem VII}
Write your own acquisition software to acquire a single satellite from the IFEN IF data file.
You can write a serial or parallel search algorithm. Provide a plot of the acquisition plane
(Code and Doppler) and provide the code phase and Doppler results. The C/A (Gold) codes
for several satellites in view are on the website.
\subsection*{Solution}
Parallel acquisition was used.  The code shift is 20 chips and the doppler frequency is 990 Hz on PRN \#7
\begin{figure}[H]
    \centering
    \includegraphics[width=0.75\linewidth]{../figures/p7_prn7.png}\label{fig:p7_prn7}
    \caption{PRN \#7 Acquisition}
\end{figure}

\section*{Appendix A: Part I Code}
\lstinputlisting[
frame=single,
numbers=left,
style=Matlab-bw
]{../matlab/p1.m}

\section*{Appendix B: Part II Code}
\lstinputlisting[
frame=single,
numbers=left,
style=Matlab-bw
]{../matlab/p2.m}

\section*{Appendix C: Part III Code}
\lstinputlisting[
frame=single,
numbers=left,
style=Matlab-bw
]{../matlab/p3.m}

\section*{Appendix D: Part IV Code}
\lstinputlisting[
frame=single,
numbers=left,
style=Matlab-bw
]{../matlab/p4.m}

\section*{Appendix E: Part VI Code}
\lstinputlisting[
frame=single,
numbers=left,
style=Matlab-bw
]{../matlab/p6.m}

\section*{Appendix C: Part VII Code}
\lstinputlisting[
frame=single,
numbers=left,
style=Matlab-bw
]{../matlab/p7.m}

\end{document}